\documentclass[11pt]{article}
\usepackage[utf8]{inputenc}
\usepackage[T1]{fontenc}
\usepackage[margin=1in]{geometry}
\usepackage{hyperref}
\usepackage{fvextra}
\usepackage{newunicodechar}
\fvset{fontsize=\footnotesize,baselinestretch=1,breaklines=true,breakanywhere=true}
\newunicodechar{⟦}{[}
\newunicodechar{⟧}{]}

\title{Latest Run – Hallucination Notes}
\author{}
\date{\today}

\begin{document}
\maketitle
\tableofcontents

\section{Overview}
I ran the new end-to-end workflow (canonicalisation, comprehensive anchor classification, halo snippet assembly, and the dg\_simple\_v2 pricing extractor) on three high-confidence agreements (Wyeth, ADESA California, and the CL\&P multi-borrower facility). Two of the outputs exhibit hallucinations or schema mismatches. This note captures the raw artifacts alongside the relevant contract passages so we can adjust the dg\_simple\_v2 prompt.

\section{Wyeth 364-Day Credit Agreement}\label{sec:wyeth}
Pricing mechanism: \textbf{ratings-only}. The contract’s “Category A–E” grid (anchors \hyperref[wyeth_anchor_evidence]{\texttt{s000013}--\texttt{s000033}}) ties the applicable margin to S\&P/Moody’s ratings; no leverage, EBITDA, or other financial ratios ever appear. The dg\_simple\_v2 output incorrectly introduces a Compustat-based \texttt{TotalDebtToEBITDA} metric and synthetic tier scheme.
\subsection{Model Output (dg\_simple\_v2)}
\VerbatimInput{runs/sanity_checks/wyeth/pricing.json}

\subsection{Contract Evidence}
\label{wyeth_anchor_evidence}
\VerbatimInput{runs/sanity_checks/wyeth/evidence.txt}

\subsection{Terminology Scan}
\VerbatimInput{runs/sanity_checks/wyeth/search_checks.txt}

\subsection{Issue}
The output invents a \texttt{TotalDebtToEBITDA} metric, a Compustat-derived formula, and a financial-metric tier scheme. The Wyeth agreement only uses ratings-based ``category A--E'' periods driven by S\&P/Moody's levels (anchors \texttt{s000013}--\texttt{s000033}) and never mentions ``Total Debt,'' ``EBITDA,'' or leverage ratios at all (confirmed by the searches above). The current dg\_simple\_v2 instructions encourage the LLM to synthesise Compustat ratios even when no such trigger exists.

\section{CL\&P / FirstLight Borrowers Credit Agreement}\label{sec:clp}
Pricing mechanism: \textbf{Eurodollar/base-rate, ratings-based}. The actual agreement quotes Eurodollar and base-rate spreads determined solely by the borrower’s public ratings (anchors \hyperref[clp_anchor_evidence]{\texttt{s000006}--\texttt{s000010}}). dg\_simple\_v2 rewrites the Eurodollar grid as a Term SOFR product even though “SOFR” never appears in the source.
\subsection{Model Output (dg\_simple\_v2)}
\VerbatimInput{runs/sanity_checks/clp/pricing.json}

\subsection{Contract Evidence}
\label{clp_anchor_evidence}
\VerbatimInput{runs/sanity_checks/clp/evidence.txt}

\subsection{Base-Rate Terminology Check}
\VerbatimInput{runs/sanity_checks/clp/search_checks.txt}

\subsection{Issue}
The contract (anchors \texttt{s000006}--\texttt{s000010}) defines pricing solely off the borrowers' public ratings, quoting ``Eurodollar rate'' and ``base rate'' advances. The dg\_simple\_v2 output, however, labels the eurodollar margin as a Term SOFR product with \texttt{base\_rate\_type} set to \texttt{"sofr"}. No ``SOFR'' string appears anywhere in the prompt view, while repeated references to ``Eurodollar'' do. This shows the prompt is also willing to swap benchmark names to modern defaults, which is incorrect for legacy agreements.

\section{Takeaways}
\begin{itemize}
  \item The dg\_simple\_v2 schema currently incentivises the model to hallucinate Compustat-friendly metrics even when the pricing grid is clearly ratings-based. The prompt needs to emphasise ``use the metrics actually cited in the supplied text; do not invent leverage/coverage ratios if they are absent.''
  \item Benchmark names must be copied verbatim. A ratings-based Eurodollar grid from 2003 should not be rewritten as a SOFR grid. The instructions should forbid benchmark substitution and require quoting the surrounding text when assigning \texttt{base\_rate} fields.
\end{itemize}

\end{document}
